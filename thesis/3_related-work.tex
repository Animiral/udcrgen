
\chapter{Related Work}
\label{chp:related-work}

The aforementioned circle packing theorem~\cite{Koebe1936} should be considered as a fundamental result from which various research into contact graphs derives. We can categorize the topic by different aspects. Consider as an exemplary overview:

\begin{itemize}
    \item by graph class: planar, outerplanar, tree, lobster, caterpillar, ...
    \item by embeddedness: free (only the graph itself is given), embedded (an embedding is prescribed)
    \item by shape: disk, (rectilinear) polygon and many more
    \item by notion of contact: intersecting, strict, weak
    \item by problem/question and associated complexity
\end{itemize}

% Felsner~\cite{Felsner2013} describes conditions and an algorithm for rectangular contact.

Just to illustrate some of the topical variations being researched, consider these examples. Any 4-connected, internally triangulated graph admits a contact representation using rectangles~\cite{kozminski_rectangular_1985}. For some shapes, like rectilinear polygons, the problem is not so much in the representation of any planar graph under the shape constraints. Instead, we want to know the maximum complexity of the shape that is necessary to represent the graph. If the graph is planar triangulated, it admits a weighed contact representation using rectilinear polygons with at most 8 vertices: ``T-shapes''~\cite{alam_computing_2013}.

Breu and Kirkpatrick~\cite{Breu1998} showed the first complexity result that specifically applies to unit disk graphs. They introduce the topic by showing that recognition of \emph{intersection} graphs, in which adjacent disks' interiors may overlap, is \NP-hard. They also state that their results extend to unit disk contact graphs, which are thus also \NP-hard to recognize in general. Hardness of the recognition problem on intersection graphs does not stop in \NP. McDiarmid and Müller~\cite{mcdiarmid_integer_2013} proved that there exist unit disk intersection graphs which do not have a polynomial certificate---required in \NP---due to the exponential number of bits required to represent an embedding function for them. The problem is complete for the class $\exists \reals$ (existential theory of reals), even harder than \NP.

On the other hand, a previous result by Klemz et al.~\cite{Klemz2015} that the recognition problem for caterpillars under the strict contact notion can be decided in linear time has since been weakened to a conjecture due to a missing proof for one of its steps~\cite{klemz_recognizing_2022}. This result is based on an observation that if two inner spine vertices of degree 5 are not separated by a spine vertex with degree at most 3 between them, the leaves are simply ``too many'' and will not fit. It is then quite easy to check whether or not a particular caterpillar has this property.

With the findings of Cleve~\cite{Cleve2020}, the author turns his attention to the weak contact notion and the triangular grid. Under this notion, caterpillars can definitely be recognized in polynomial time\footnote{Although the original source claims linear time, this proof has since been retracted. Instead, caterpillars can be recognized in quadratic time at worst (Jonas Cleve, personal communication to Soeren Terziadis).}, and the recognition problem remains \NP-hard for trees.

We continue from these results with this thesis, going hand in hand with a new paper by Bhore et al.~\cite{Bhore2021}. They take the next step from the caterpillar to the slightly more complex lobster. Our particular attention concerns the proof that, assuming the x-monotonicity and triangular grid layout, these graphs can also be recognized in linear time. The method is more difficult than for caterpillars, as it requires a constructive algorithm with a dynamic programming approach---the same approach which we discuss in Chapter~\ref{ch:dynprog} and implement.

Interestingly, the recognition problem becomes harder if an embedding is prescribed in addition to the graph itself. It is even \NP-hard to recognize embedded unit disk contact caterpillars under weak contact~\cite{Chiu2019}. The given embedding robs us of our choice to place the subtrees such that the spine cannot ``turn in on itself''. While the complexity of strict UDC for trees is yet unknown, if the tree is embedded, it remains \NP-hard~\cite{Bowen2015}.
